\section{Conclusión y trabajo futuro} \label{sec:conclusion}

La aplicación presentada en este trabajo se basa en un nuevo modelo ontológico para representar tareas agropecuarias, que puede ser aplicado en un dominio amplio y a múltiples casos de uso. El desarrollo consiste en una misma base de código que se puede compilar y distribuir para múltiples plataformas: web, PWA, escritorio y móvil.

Fueron propuestas una serie de características que diferencian este sistema respecto de las demás alternativas disponibles en el mercado: por un lado, privacidad y anonimato dado que no se requiere registro ni autenticación, soporte \textit{offline} prolongado y mayor control por parte del usuario en el manejo de las herramientas de cálculo y consulta de información.

El software posibilita también el intercambio de conocimiento representado mediante bases de datos y herramientas de cálculo, lo que potencialmente podría significar la generación de espacios para el comercio digital o \textit{marketplaces} de asesoramiento, información y conocimiento en general. Sin embargo, este esquema distribuido requiere de una red de nodos numerosa para garantizar una disponibilidad de datos aceptable, o bien el desarrollo de mecanismos para la ejecución de los nodos en segundo plano.

Entre los desafíos por resolver resta implementar la navegación y búsqueda de las bases de datos disponibles en la red, ya que, como se mencionó anteriormente, la descarga de una base de datos se realiza a partir del código identificador o CID que se debe conocer de antemano.

Finalmente, resta evaluar las limitaciones del sistema en función de la escala, lo que eventualmente requerirá de la optimización de algunos procesos críticos.
