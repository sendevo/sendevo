\section{Introducción}

Durante el transcurso de los últimos años, el advenimiento de la inteligencia artificial (AI, por sus siglas en inglés) está revolucionando el desarrollo de productos tecnológicos en múltiples ámbitos. En el contexto de la producción agropecuaria, una de las potenciales aplicaciones de la AI es la automatización de la toma de decisiones, la cual, en un sentido amplio, es un desafío que aún sigue sin resolverse \cite{ulitin2019,bao2022,schmitt2023}. Esta tarea requiere de un volumen de datos considerable, los cuales deben estar bien estructurados y organizados para garantizar la eficacia del modelo \cite{beleites2013,figueroa2012}. 

Por esta razón es que resulta fundamental contar con un formato de datos que permita modelar las tareas agropecuarias, el momento y el lugar en que se realizan, así como también los motivos por los cuales se ejecutan, como por ejemplo, las condiciones climáticas, valores de mercado, disponibilidad de insumos, etc. Si bien se han desarrollado distintos modelos ontológicos para formalizar conceptos en el dominio de las tareas agropecuarias \cite{abrahao2017,abrahao2018}, no se han validado estas propuestas mediante el desarrollo de aplicaciones de software. Uno de los aportes de este trabajo consiste en la propuesta de un formato para modelar y describir digitalmente las características de distintas tareas que se realizan en el contexto de la producción agropecuaria. 

Una vez contando con un formato adecuado y flexible para representar un conjunto diverso de eventos que ocurren en el contexto agroproductivo, sólo resta avanzar con el ingreso de datos, es decir la populación de una base de datos lo suficientemente extensa a partir de la cual sea posible extraer información útil y entrenar modelos de aprendizaje automático que asistan a la toma de decisiones. Para lograr este objetivo se evidencia la necesidad de contar con un gran número de usuarios, una tarea que implica ofrecer un producto atractivo a un mercado digital competitivo y en constante crecimiento, lo cual representa un desafío considerable. 

En la actualidad, las principales plataformas digitales con las que es posible registrar o documentar tareas agropecuarias, poseen algunas características que imponen cierta \textit{fricción} al usuario \cite{encuesta}, es decir, limitaciones que previenen a los usuarios de lograr ciertos objetivos o tareas con facilidad mediante el uso del sistema. La característica más recurrente es el requerimiento del registro o identificación del usuario, que atenta contra la privacidad del mismo, al tener que revelar algunos datos personales y configurar factores de autenticación. Otra limitación que se presenta en muchos sistemas, es el tiempo que se debe destinar al ingreso de datos, por ejemplo, de campos, lotes, maquinarias, animales, etc. antes de poder ejecutar cualquier acción, como generar órdenes de labor, ejecutar análisis con herramientas propias del sistema o compartir información con otros usuarios.

Con el objetivo de resolver, al menos en parte, las problemáticas expuestas anteriormente, se procedió con el diseño e implementación de un software multiplataforma que, operando con un sistema de archivos descentralizado, ofrece al usuario soporte offline prolongado, una base de datos para el registro y consulta de tareas agropecuarias, mecanismos de edición colaborativa de datos y múltiples herramientas de cálculo. Se presentan en este artículo, los requerimientos planteados, la arquitectura propuesta y los detalles más relevantes acerca del funcionamiento del sistema. 

El resto del trabajo se organiza de la siguiente manera: en la sección \ref{sec:modelo} se describe el modelo ontológico empleado para representar a las tareas agropecuarias, se proponen dos ejemplos ilustrativos y se discuten distintos casos de uso. En la sección \ref{sec:datos} se presenta el sistema de archivos que se propone y la relación entre las distintas bases de datos con las que funciona el sistema. En la sección \ref{sec:software} se discute la elección de la tecnología y se detallan las cuestiones técnicas involucradas en la implementación y el despliegue del software para múltiples plataformas. Finalmente, en la sección \ref{sec:conclusion} se concluye el trabajo, se enumeran los avances pendientes y posibles futuros pasos. 